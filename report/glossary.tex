\makeglossaries
% \makenoidxglossaries  % slower, but no need to do $ makeglossaries report.tex

% terms

\newglossaryentry{assembly} {
  name = {assembly language},
  description = {A type of low-level programming language intended to communicate directly with the computer's hardware}
}

\newglossaryentry{assembly simulator} {
  name = {assembly simulator},
  description = {}
  see = {assembly}
}

\newglossaryentry{computer} {
  name = {computer},
  description = {}
}

\newglossaryentry{programming} {
  name = {programming},
  description = {The process of expressing abstract ideas and models within a language}
}

\newglossaryentry{programmer} {
  name = {programmer},
  description = {A person who programs},
  see = {programming}
}

\newglossaryentry{programming language} {
  name = {programming language},
  description = {},
}

\newglossaryentry{program} {
  name = {program},
  description = {},
  see = {programming}
}

\newglossaryentry{high-level language} {
  name = {high-level language},
  description = {A programming language with higher level abstractions, closer to the human language},
  see = {programming language}
}

\newglossaryentry{low-level language} {
  name = {low-level language},
  description = {A programming language with low level abstractions, closer to machine instructions}
  see={programming language}
}

\newglossaryentry{compilation} {
  name = {compilation},
  description = {}
}

\newglossaryentry{computer science} {
  name = {computer science},
  description = {}
}

\newglossaryentry{software engineer} {
  name = {software engineer},
  description = {}
}

\newglossaryentry{processor} {
  name = {processor},
  description = {}
}

\newglossaryentry{debugging} {
  name = {debugging},
  description = {}
}

\newglossaryentry{lisp} {
  name = {LISP},
  description = {}
}

\newglossaryentry{clock cycle} {
  name = {clock cycle},
  description = {}
}

\newglossaryentry{memory} {
  name = {memory},
  description = {}
}

\newglossaryentry{data memory} {
  name = {data memory},
  description = {},
  see = {memory}
}

\newglossaryentry{text memory} {
  name = {text memory},
  description = {},
  see = {memory}
}

\newglossaryentry{register} {
  name = {register},
  description = {}
}

\newglossaryentry{stack} {
  name = {stack},
  description = {}
}

\newglossaryentry{desktop device} {
  name = {desktop device},
  description = {}
}

\newglossaryentry{embedded device} {
  name = {embedded device},
  description = {}
}

\newglossaryentry{web application} {
  name = {web application},
  description = {}
}

\newglossaryentry{port} {
  name = {port},
  description = {}
}

\newglossaryentry{pipeline} {
  name = {pipeline},
  description = {}
}

\newglossaryentry{instruction} {
  name={instruction},
  description={}
}

\newglossaryentry{instruction dependency} {
  name={instruction dependency},
  description={},
  see={instruction}
}

\newglossaryentry{pseudo-instruction} {
  name={pseudo-instruction},
  description={},
  see={instruction}
}

\newglossaryentry{directive} {
  name={directive},
  description={}
}

\newglossaryentry{pipeline stalls} {
  name = {pipeline stalls},
  description = {},
  see = {pipeline}
}

\newglossaryentry{RISC-V} {
  name = {RISC-V},
  description = {}
}

\newglossaryentry{MIPS} {
  name = {MIPS},
  description = {}
}

\newglossaryentry{ARM} {
  name = {ARM},
  description = {}
}

\newglossaryentry{debug} {
  name = {debug},
  description = {}
}

\newglossaryentry{obfuscate} {
  name = {obfuscate},
  description = {}
}

\newglossaryentry{subroutine} {
  name = {subroutine},
  description = {}
}

\newglossaryentry{theorem prover} {
  name = {theorem prover},
  description = {A computer program or tool used in the field of logic and mathematics to automatically verify the validity of mathematical statements, propositions, or theorems}
}

\newglossaryentry{branch} {
  name = {branch},
  description = {}
}

\newglossaryentry{arithmetic operation} {
  name = {arithmetic operation},
  description = {}
}

\newglossaryentry{logical operation} {
  name = {logical operation},
  description = {}
}

\newglossaryentry{web service} {
  name = {web service},
  description = {}
}

\newglossaryentry{interpreter} {
  name = {interpreter},
  description = {}
}

\newglossaryentry{programming paradigm} {
  name = {programming paradigm},
  description = {}
}

\newglossaryentry{program counter} {
  name = {program counter},
  description = {}
}

\newglossaryentry{word} {
  name = {word},
  description = {}
}

\newglossaryentry{REPL environment} {
  name = {REPL environment},
  description = {}
}


% acronyms
\newglossaryentrywithacronym{FOSS}
{Free Open Source Software}
{A type of software that is free to use and has its source code public}

\newglossaryentrywithacronym{CLI}
{Command Line Interface}
{}

\newglossaryentrywithacronym{GUI}
{Graphical User Interface}
{}

\newglossaryentrywithacronym{UI}
{User Interface}
{}


\newglossaryentrywithacronym{I/O}
{Input/Output}
{}

\newglossaryentrywithacronym{RISC}
{Reduced Instruction Set Computing}
{}

\newglossaryentrywithacronym{ISA}
{Instruction Set Architecture}
{}

\newglossaryentrywithacronym{ALU}
{Arithmetic Logic Unit}
{}

\newglossaryentrywithacronym{AST}
{Abstract Syntax Tree}
{}

\newglossaryentrywithacronym{HTTP}
{HyperText Transfer Protocol}
{}

\newglossaryentrywithacronym{CPU}
{Central Processing Unit}
{}

\newglossaryentrywithacronym{STDOUT}
{Standard Output}
{}

\newglossaryentrywithacronym{STDIN}
{Standard Input}
{}

\newacronym{json}{JSON}{JavaScript Object Notation}

\newacronym{yaml}{YAML}{YAML Ain't Markup Language}
