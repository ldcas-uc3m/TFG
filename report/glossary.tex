\makeglossaries
% \makenoidxglossaries  % slower, but no need to do $ makeglossaries report.tex

% terms

\newglossaryentry{assembly} {
  name = {assembly language},
  description = {A type of low-level programming language intended to communicate directly with the computer's hardware}
}

\newglossaryentry{assembly simulator} {
  name = {assembly simulator},
  description = {A piece of software that simulates a computer that executes an assembly language}
  see = {assembly}
}

\newglossaryentry{computer} {
  name = {computer},
  description = {A machine that can be programmed to automatically carry out sequences of arithmetic or logical operations}
}

\newglossaryentry{programming} {
  name = {programming},
  description = {The process of expressing abstract ideas and models within a language}
}

\newglossaryentry{programmer} {
  name = {programmer},
  description = {A person who programs},
  see = {programming}
}

\newglossaryentry{programming language} {
  name = {programming language},
  description = {A system of notation for expressing computer programs},
  see = {program}
}

\newglossaryentry{program} {
  name = {program},
  description = {A sequence or set of instructions for a computer to execute},
  see = {programming}
}

\newglossaryentry{high-level language} {
  name = {high-level language},
  description = {A programming language with higher level abstractions, closer to the human language},
  see = {programming language}
}

\newglossaryentry{low-level language} {
  name = {low-level language},
  description = {A programming language with low level abstractions, closer to machine instructions}
  see={programming language}
}

\newglossaryentry{compilation} {
  name = {compilation},
  description = {The process of translating computer programs from one programming language into another, typically from a high into a low-level language}
}

\newglossaryentry{computer science} {
  name = {computer science},
  description = {The study of computation, information, and automation}
}

\newglossaryentry{software engineering} {
  name = {software engineer},
  description = {An engineering approach to software development, which involves the definition, implementation, testing, management and maintenance of software systems}
}

\newglossaryentry{processor} {
  name = {processor},
  description = {An electrical component that executes instructions and performs operations on external data sources}
}

\newglossaryentry{debugging} {
  name = {debugging},
  description = {The process of finding the cause of, and solving, undesired behavior in computer programs}
}

\newglossaryentry{clock cycle} {
  name = {clock cycle},
  description = {A single increment of a processor's clock, during which the smallest unit of activity is carried out}
}

\newglossaryentry{memory} {
  name = {memory},
  description = {A digital storage device used to store information}
}

\newglossaryentry{data memory} {
  name = {data memory},
  description = {A specialized type of memory used to store exclusively program data},
  see = {memory}
}

\newglossaryentry{text memory} {
  name = {text memory},
  description = {A specialized type of memory used to store exclusively program instructions},
  see = {memory}
}

\newglossaryentry{register} {
  name = {register},
  description = {A small data storage inside a computer's processor, used for storing data to operate}
}

\newglossaryentry{stack} {
  name = {stack},
  description = {A portion of memory in which the information or item stored last is retrieved first, typically used to store the local variables of a subroutine when it is called}
}

\newglossaryentry{desktop device} {
  name = {desktop device},
  description = {A personal computer designed for regular use}
}

\newglossaryentry{embedded device} {
  name = {embedded device},
  description = {An independent device responsible for executing a specific task within a larger system}
}

\newglossaryentry{web application} {
  name = {web application},
  description = {An application software that is accessed using a web browser}
}

\newglossaryentry{port} {
  name = {port},
  description = {A connection endpoint to direct data to a specific service or application}
}

\newglossaryentry{pipeline} {
  name = {pipeline},
  description = {A technique for implementing simultaneous execution of a sequence of instructions within a single processor}
}

\newglossaryentry{instruction} {
  name={instruction},
  description={An order given to a processor that makes the computer take some action}
}

\newglossaryentry{instruction dependency} {
  name={instruction dependency},
  description={A situation in which an instruction refers to the data of a preceding statement},
  see={instruction}
}

\newglossaryentry{pseudo-instruction} {
  name={pseudo-instruction},
  description={In an ISA, an instruction that is internally divided into several instructions by the processor},
  see={instruction}
}

\newglossaryentry{directive} {
  name={assembly directive},
  description={Directions for the assembler to perform some specific action or change a specific setting}
}

\newglossaryentry{pipeline stalls} {
  name = {pipeline stalls},
  description = {A delay in execution of an instruction in order to resolve a hazard},
  see = {pipeline}
}


\newglossaryentry{obfuscate} {
  name = {obfuscate},
  description = {The act of creating source or machine code that is difficult for humans or computers to understand}
}

\newglossaryentry{subroutine} {
  name = {subroutine},
  description = {A callable unit of software logic that has a well-defined interface and behavior and can be invoked multiple times}
}

\newglossaryentry{theorem prover} {
  name = {theorem prover},
  description = {A computer program or tool used in the field of logic and mathematics to automatically verify the validity of mathematical statements, propositions, or theorems}
}

\newglossaryentry{branch} {
  name = {branch},
  description = {The act of switching execution to a different instruction sequence as a result of evaluating an expression}
}

\newglossaryentry{arithmetic operation} {
  name = {arithmetic operation},
  description = {A mathematical operation applied to numeric values}
}

\newglossaryentry{logical operation} {
  name = {logical operation},
  description = {A mathematical operation applied to logical values}
}

\newglossaryentry{web service} {
  name = {web service},
  description = {A service offered by an electronic device to another electronic device, communicating with each other via the Internet}
}

\newglossaryentry{interpreter} {
  name = {interpreter},
  description = {A computer program that directly executes instructions written in a programming or scripting language}
}

\newglossaryentry{programming paradigm} {
  name = {programming paradigm},
  description = {A method to solve a problem using tools and techniques that are available to us following some abstract approach}
}

\newglossaryentry{program counter} {
  name = {program counter},
  description = {A processor register that points to the next instruction to be executed}
}

\newglossaryentry{word} {
  name = {word},
  description = {A fixed-sized datum handled as a unit by the instruction set}
}

\newglossaryentry{REPL environment} {
  name = {REPL environment},
  description = {A simple interactive computer programming environment that reads a user input, executes it, and returns the result to the user}
}

\newglossaryentry{system call} {
  name = {system call},
  description = {The programmatic way in which a computer program requests a service from the operating system on which it is executed}
}

\newglossaryentry{build system} {
  name = {build system},
  description = {A computer program that orchestrates the compilation of a software system}
}

\newglossaryentry{cloud computing} {
  name = {cloud computing},
  description = {The on-demand availability of computer system resources without direct active management by the user}
}


% acronyms
\newglossaryentrywithacronym{FOSS}
{Free Open Source Software}
{A type of software that is free to use and has its source code public}

\newglossaryentrywithacronym{CLI}
{Command Line Interface}
{A way of interacting with a computer program by inputting lines of text}

\newglossaryentrywithacronym{GUI}
{Graphical User Interface}
{A form of user interface that allows users to interact with electronic devices through graphical icons and visual indicators}

\newglossaryentrywithacronym{UI}
{User Interface}
{The space where interactions between humans and machines occur}


\newglossaryentrywithacronym{I/O}
{Input/Output}
{The communication between a computer and the outside world}

\newglossaryentrywithacronym{RISC}
{Reduced Instruction Set Computing}
{A computer architecture designed to simplify the individual instructions given to the computer to accomplish tasks}

\newglossaryentrywithacronym{ISA}
{Instruction Set Architecture}
{The set of instructions that a computer processor can understand and execute, as well as an abstract model of the architecture of a processor}

\newglossaryentrywithacronym{ALU}
{Arithmetic Logic Unit}
{The part of the processor that carries out the different operations}

\newglossaryentrywithacronym{AST}
{Abstract Syntax Tree}
{A data structure used to represent the structure of a program or instruction}

\newglossaryentrywithacronym{HTTP}
{HyperText Transfer Protocol}
{An protocol designed to transfer information between networked devices}

\newglossaryentrywithacronym{CPU}
{Central Processing Unit}
{The main processor in a computer, in charge of decoding and executing instructions}

\newglossaryentrywithacronym{STDOUT}
{Standard Output}
{The default file descriptor where a process can write output}

\newglossaryentrywithacronym{STDIN}
{Standard Input}
{The default input device that the program uses to read data}

\newglossaryentrywithacronym{RAM}
{Random Access Memory}
{A form of computer memory that can be read and changed in any order, but requires power to retain the data}

\newglossaryentrywithacronym{IoT}
{Internet of Things}
{A network of interrelated devices that connect and exchange data with other IoT devices and the cloud}

\newglossaryentrywithacronym{AI}
{Artificial Inteligence}
{A technology that enables computers to simulate human intelligence and problem-solving capabilities}


\newacronym{json}{JSON}{JavaScript Object Notation}

\newacronym{yaml}{YAML}{YAML Ain't Markup Language}

\newacronym{gplv3}{GPLv3}{GNU General Public License, version 3}
