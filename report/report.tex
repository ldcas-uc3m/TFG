\documentclass[en]{uc3mthesisIEEE}


\usepackage{import}
\usepackage{lipsum}  % dummy text

\usepackage{mymacros}  % report-specific macros


% silence ht warnings
\usepackage{silence}
\WarningFilter{latex}{`h' float specifier changed to `ht'}


% REFERENCES
\addbibresource{references.bib}  % bibliography file
\import{}{glossary.tex}  % glossary file


%	DOCUMENT

% setup
\degree{Bachelor's degree in Computer Science and Engineering}
\course{2023-2024}
\title{A Didactic and Generic Assembly Language Simulator}
\author{Luis Daniel Casais Mezquida}
\tutors{Alejandro Calderón Mateos}
\place{Leganés, Madrid, Spain}
\date{June 2024}

\begin{document}

  % COVER
  \makecover

  % EPIGRAPH
  \makeepigraph
    {Nor again is there anyone who loves or pursues or desires to obtain pain of itself, because it is pain, but because occasionally circumstances occur in which toil and pain can procure him some great pleasure}  % quote
    {Cicero}  % author
    {de Finibus Bonorum et Malorum}  % source


  % ABSTRACT
  \begin{abstract}
    \lipsum[1-3]
    \keywords{Caca, culo, pedo, pis}
  \end{abstract}


  % ACKNOWLEDGEMENTS
  \begin{acknowledgements}
    \lipsum[1]
  \end{acknowledgements}


  % TOC
  \tableofcontents
  \listoffigures
  \listoftables


  % THESIS
  \begin{thesis}
    \includefrom{parts/}{introduction.tex}
    \includefrom{parts/}{state_of_the_art.tex}
    \includefrom{parts/}{analysis.tex}
    \includefrom{parts/}{design.tex}
    \includefrom{parts/}{implementation.tex}
    \includefrom{parts/}{evaluation.tex}
    \includefrom{parts/}{planning.tex}
    \includefrom{parts/}{conclusions.tex}
    % \newpage
    % \import{parts/}{example.tex}
  \end{thesis}

  % BIBLIOGRAPHY
  \cleardoublepage
  \label{bibliography}
  \printbibliography[heading=bibintoc]

  % GLOSSARY
  \cleardoublepage
  \label{glossary}
	\printglossaries
	% \printnoidxglossaries[type=\acronymtype]  % slower, but no need to do $ makeglossaries report


  % APPENDICES
  % \begin{appendices}
  %   \chapter{My stuff}
  %   \lipsum
  % \end{appendices}


\end{document}