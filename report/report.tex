%!TEX program = lualatex
\documentclass[en]{uc3mthesisIEEE}

\usepackage{import}  % better than \input and \include
\usepackage{enumitem}  % control item separation -> \begin{itemize}[nosep, noitemsep]
\usepackage{lipsum}  % dummy text
\usepackage{placeins}  % \FloatBarrier
\usepackage[most]{tcolorbox}  % beautiful boxes
\usepackage[nounderscore]{syntax}  % BNF grammars
\usepackage{dirtree}  % directory trees
\usepackage{pdflscape}  % landscape environment
\usepackage{pgfgantt}  % gannt diagrams


\usepackage[en,enableTraceability,enableCaptions]{SRS}  % requirements
\usepackage{mymacros}  % report-specific macros


% configure syntax's grammar environment
\setlength{\grammarparsep}{8pt} % separation between rules
\setlength{\grammarindent}{8em} % separation between LHS/RHS

% repository URL
\newcommand{\myrepo}{\url{https://github.com/ldcas-uc3m/TFG}}


% silence ht warnings
\usepackage{silence}
\WarningFilter{latex}{`h' float specifier changed to `ht'}


% REFERENCES
\addbibresource{references.bib}  % bibliography file
\import{}{glossary.tex}  % glossary file


%	DOCUMENT

% setup
\degree{Bachelor's degree in Computer Science and Engineering}
\course{2023-2024}
\title{Analysis, Design and Implementation of a Didactic and Generic Assembly Language Simulator}
\shorttitle{AD\&I of a Didactic and Generic Assembly Language Simulator}
\author{Luis Daniel Casais Mezquida}
\tutors{Alejandro Calderón Mateos}
\place{Leganés, Madrid, Spain}
\date{June 2024}


\begin{document}

  % COVER
  \makecover


  % EPIGRAPH
  \makeepigraph
    {
      % this perfectly summarizes my experience with LaTeX, C++, and CMake
      Nor again is there anyone who loves or pursues or desires to obtain pain of itself, because it is pain, but because occasionally circumstances occur in which toil and pain can procure him some great pleasure.
    }  % quote
    {Cicero}  % author
    {de Finibus Bonorum et Malorum}  % source


  % ACKNOWLEDGEMENTS
  \begin{acknowledgements}
    % parents
    I would like to start by thanking my parents. It is thanks to them that I managed to be here right now, writing my bachelor's thesis. Not only because they ``paid for the damned thing'' (and more than they should've, thanks to Algebra and Physics), but because they supported, educated, and guided me through my whole life. I wouldn't be the person that I am today without them.

    % friends & gf
    The rest of the support has come from my friends, and specially my girlfriend. Here, I make a rather liberal use of the word `support', as it has mainly consisted of me yapping about ``computer stuff'' and them nodding politely. I will probably never know what (low) percentage of the things I have told them are still in their memory, but I'm thankful to them for entertaining my love for teaching and sharing knowledge. I know they love me, and I love them too.

    % tutor, @joseaverde
    A special consideration is deserved for the two people that helped make this project (me finally graduating) possible: my tutor, Alex, and Jose Antonio.

    I had to `stalk' Alex to meet with him the first time (and some subsequent times), but he was generous enough not only to accept me, but also to propose ideas for the thesis (as I had none). Despite his \textit{very} full schedule, he accepted scheduling weekly tutoring sessions for the report, as the deadline was rapidly approaching, and I was very lost.

    Jose Antonio has been the greatest help for me, despite being a junior student. He has helped me navigate the C++ language, which is not an easy task, and was a great partner to share the frustrations of this great language. He also contributed some ideas to the project, such as using a Lisp-like language to define the instructions. I think we can all expect great things from you.

    I'd also like to mention professor Jose Daniel, who introduced me to the world of C++ and pointed me to Alex for project ideas, after I initially asked him.

    % GUL
    Another group of people that deserves an acknowledgement is \textit{el} GUL. I entered the association two years ago, but I am now its president. I have to be thankful not only of the people, Danié, Banga, Alex, Jose Antonio (again), Jorge, Cía, and the rest of members, for teaching me a lot of extremely useful stuff, transforming me into a Linux believer, and keeping me company; but also for offering me the opportunity of organizing and presenting talks and workshops, and sharing my knowledge.

    % professors, colleagues
    There are many other people that got me through the degree, even if their contributions were small. This goes for all of my excellent professors, who gave me the tools that I will surely make use of in the future, and my colleagues, who helped and inspired me throughout the years. This is getting too long, so I hope they'll forgive me for not explicitly mentioning them.

    % open-source community
    Finally, I'd like to thank the open-source community. Most of my involvement with them may have been passive, using excellent software and reading documentation, tutorials, and GitHub issues, but I have to admire their dedication for making the world a better place, one line of code at a time. I'll be sure to follow their steps.
  \end{acknowledgements}


  % ABSTRACT
  \begin{abstract}
    There are not many didactic and generic assembly language simulators that can be used to introduce people to assembly language programming and instruction Set Architecture (ISA) design.

    Due to the current geopolitical and commercial situation, countries and corporations have started investing in the manufacturation of their own chips, and open and modular standards such RISC-V act as a base for designing and implementing new ISAs.

    Our proposal aims to foment ISA design and assembly programming, by providing a simple and easy to understand tool for anyone to start exploring different instruction sets, while also aiding people with the sometimes difficult process of learning assembly language programming.

    This project presents a simple, open-source, and intuitive simulator, focused on providing the user an intuitive knowledge of ISA design and assembly programming. Our simulator allows users to create their own ISAs, by defining instructions in a simple Lisp-like language, and to interpret any assembly-like program using that definition.

    \keywords{Simulation, assembly, ISA, interpreter}
  \end{abstract}


  % TOC
  \tableofcontents
  \listoffigures
  \listoftables


  % THESIS
  \begin{thesis}
    \includefrom{parts/}{introduction.tex}
    \includefrom{parts/}{state_of_the_art.tex}
    \includefrom{parts/}{analysis.tex}
    \includefrom{parts/}{design.tex}
    \includefrom{parts/}{implementation.tex}
    \includefrom{parts/}{validation.tex}
    \includefrom{parts/}{plan.tex}
    \includefrom{parts/}{conclusions.tex}
  \end{thesis}


  % BIBLIOGRAPHY
  \cleardoublepage
  \label{bibliography}
  \printbibliography[heading=bibintoc,title={References}]


  % GLOSSARY
  \cleardoublepage
  \label{glossary}
	\printglossaries
	% \printnoidxglossaries[type=\acronymtype]  % slower, but no need to do $ makeglossaries report


  % APPENDICES
  \begin{appendices}
    \includefrom{parts/}{manual.tex}
  \end{appendices}


\end{document}