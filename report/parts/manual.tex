\chapter{User manual}\label{app:manual}

\section{Architecture definition file}
The structure of the \gls{json} files is the following:
\begin{itemize}[nosep,noitemsep]
  \item \texttt{name}: Name of the architecture.
  \item \texttt{memory}: Memory configuration.
    \begin{itemize}[nosep,noitemsep]
      \item \texttt{text}: Text memory configuration.
        \begin{itemize}[nosep,noitemsep]
          \item \texttt{start\_addr}: Memory start address (\texttt{string} - HEX).
        \end{itemize}
      \item \texttt{data}: Data memory configuration.
        \begin{itemize}[nosep,noitemsep]
          \item \texttt{start\_addr}: Memory start address (\texttt{string} - HEX).
          \item \texttt{end\_addr}: Memory end address (\texttt{string} - HEX).
        \end{itemize}
    \end{itemize}
  \item \texttt{registers}: Register names (\texttt{array[string]}).
  \item \texttt{syscalls}: System call map (\texttt{object[string, string]}).
    \begin{itemize}[nosep,noitemsep]
      \item \texttt{<id>}: \texttt{"print\_int"} | \texttt{"print\_char"} | \texttt{"read\_int"} | \texttt{"read\_char"} | \texttt{"exit"}
    \end{itemize}
  \item \texttt{instruction\_set}: Set of architecture instructions.\\
    Each instruction is a JSON object with the following format:
    \begin{itemize}[nosep,noitemsep]
      \item \texttt{<instruction\_name>}
      \begin{itemize}[nosep,noitemsep]
        \item \texttt{args}: Function arguments. They must all start by \texttt{\$}. (\texttt{array[string]})
        \item \texttt{def}: LUISP-DA definition (\texttt{string}).
      \end{itemize}
    \end{itemize}
  \item \texttt{data\_types}: Map of memory data types and their mappings.  
      Supported data types are:
      \begin{itemize}[nosep,noitemsep]
        \item \texttt{str}: zero-terminated strings
        \item \texttt{word}: 32-bit words
        \item \texttt{none}: no type
      \end{itemize}
\end{itemize}


\section{Instalation and execution}

\begin{enumerate}
  \item Download and decompress the source code from the repository (\myrepo).
  \item Move to the \texttt{build} directory and build and compile the source code:
  \begin{tcolorbox}[bottom=-8pt]
    \begin{verbatim}
cd build/
cmake -G Ninja ..
cmake --build .
    \end{verbatim}
  \end{tcolorbox}
  \item Execute the simulator (which can be found in \texttt{build/src/}) with an input \gls{ISA} file. E.g.:
  \begin{tcolorbox}[bottom=-8pt]
    \begin{verbatim}
./src/tfg --isa=../architectures/RISC-V.json
    \end{verbatim}
  \end{tcolorbox}
\end{enumerate}