\chapter{Analysis}\label{chap:analysis}
This chapter describes the proposed solution by briefly recapping the project (\sectionref{project-description}), specifying the user's requirements (\sectionref{user-requirements}), deriving the software requirements (\sectionref{soft-requirements}), and specifying the possible use cases of the system (\sectionref{use-cases}). Finally, we verify our specifications through the use of traceability matrices (\sectionref{req-traceability}).



\section{Project description}\label{sec:project-description}
% pequeño resumen del projecto



\section{User requirements}\label{sec:user-requirements}
This section provides a detailed description of the user's requirements for the project. These requirements indicate the main functionality and restrictions the developed system must fulfill.

The user requirements are divided into two distinct types:
\setlist{nosep}
\begin{itemize}[noitemsep]
  \item \textbf{\nameref{subsec:capacities}:} Describe the expected system's functionality.
  \item \textbf{\nameref{subsec:restrictions}:} Impose constraints or conditions that the system must fulfill.
\end{itemize}

Each user requirement is uniquely identified by an ID, which follows the format \textit{UR-YY-XX}, where \textit{YY} identifies the type of the requirement, either a capacity (\textit{CA}) or a restriction (\textit{RE}); and \textit{XX} identifies the sequential number of the requirement, starting at \textit{01}. Table \ref{tab:ur-template} provides the template used for the specification of the requirements, including the description of each attribute.

\printureqtemplate{UR-YY-XX}


\subsection{Capacities}\label{subsec:capacities}

% \reqref{UR-CA-01}
\begin{userReq}{UR-CA-01}{pc=m,pd=l,s=nc,v=h}
  \lipsum[1][1-3]
\end{userReq}


\subsection{Restrictions}\label{subsec:restrictions}




\section{Software requirements}\label{sec:soft-requirements}
This section provides a detailed description of the system's software requirements for the project. These requirements are derived from the user requirements, defined in Section \ref{sec:user-requirements}, and comprise the software specifications for the system.

The user requirements are divided into two distinct types:
\setlist{nosep}
\begin{itemize}[noitemsep]
  \item \textbf{\nameref{subsec:func-req}:} Specify the software's functionalities and characteristics.
  \item \textbf{\nameref{subsec:non-func-req}:} Specify other non-functional characteristics of the software.
\end{itemize}

Each user requirement is uniquely identified by an ID, which follows the format \textit{SR-YY-XX}, where \textit{YY} identifies the type of the requirement, either functional (\textit{FC}) or non-functional (\textit{NF}); and \textit{XX} identifies the sequential number of the requirement, starting at \textit{01}. Table \ref{tab:sr-template} provides the template used for the specification of the requirements, including the description of each attribute.

\printsreqtemplate{SR-YY-XX}


\subsection{Functional requirements}\label{subsec:func-req}


\subsection{Non-functional requirements}\label{subsec:non-func-req}




\section{Use cases}\label{sec:use-cases}

\printuctemplate{UC-XX}





\section{Traceability}\label{sec:req-traceability}