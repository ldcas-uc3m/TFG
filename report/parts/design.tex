\chapter{Design}\label{chap:design}



\section{Study of the solution}\label{sec:study}
% study of tools (LISP, CPP, JSON) (STATE OF THE ART II)
% - tables comparing tools

% language: cpp (ISO), rust, python, JS

% interpreter: flex/bison, ANTLR, custom (MAL)

% arch languages: lua, lisp/forth, sub-asm (IEEE 694-1985), custom

% build system: none/make, cmake, conan, crpbuild/gnat

% arch definition: JSON, YAML, custom



\section{System architecture}\label{sec:architecture}

\subsection{Components}

\printcomptemplate{Identifier}

\begin{component}{acasdf}
  {}  % role
  {}  % dependencias
  {}  % Descripción
  {in: ,out: }  % data
  {SR-FC-01}  % origen
  % descripción
\end{component}

\FloatBarrier  % prevent floats from passing this point


% UI (diagrama de flujo??)

% logica -> sequence diagram (interprete)

% JSON -> diagrama entidad-relación

% definición lenguaje LUISP-DA (backus-naur form)


\subsection{Traceability}\label{subsec:comp-traceability}
% Note that the `-` is a special character in Lua patterns and should be escaped, using \37.

% SR-NC - UR-RE
\begin{figure}[h]
  \label{tab:traceability-comp-fc}
  \centering
  \traceabilityPrintMatrix{^[^US][^R][^-]}{^SR\37-FC}{}
  \caption{Traceability between functional requirements and components}
\end{figure}