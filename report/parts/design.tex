\chapter{Design}\label{chap:design}
This chapter provides a full description of the proposed solution. It details the design process of the proposed solution, by discussing the different alternatives (\secref{study}), and describes the proposed architecture (\secref{architecture}), including all components and design decisions.


\section{Study of the solution}\label{sec:study}  % STATE OF THE ART II
After the elicitation of the requirements (Section \ref{sec:requirements}) and use cases (Section \ref{sec:use-cases}) of the system, we must discuss the different tools at our disposition to build the application, as our design will depend on them.


\subsection{Instruction definition language}\label{subsec:inst-lang-study}
\reqref[Requirement]{SR-FC-05} calls for the creation of a language to encode the functionality of the different \glspl{instruction} on the \gls{ISA}. This language is the one that the simulator will actually execute. This language, as per \reqref[requirement]{SR-NF-09}, should be simple and easy to understand.

One approach is to use a generic assembly language, such as the one described in the IEEE Standard for Microprocessor Assembly Language \parencite{IEEE-694-1985}. This is a set of common set of \glspl{instruction} for all microprocessors, and \textit{should} cover all \gls{ISA} possibilities. The main problem with this solution is that it wouldn't solve our main problem, to teach the user \gls{assembly} programming, as the \glspl{instruction} would be encoded in yet another \gls{assembly}. Therefore, we must use some other type of simple and generic language that's easier to understand.

There are scripting languages with good integration with other programming languages, such as Lua \parencite{lua}. Being an embeddable language, it can function as a simple language to interact with an application. Lua is currently used in several projects, such as Neovim \parencite{nvim-lua} and Lua\TeX~\parencite{luatex}. The problem with this approach is \reqref[requirement]{SR-NF-08}; Lua is not easily validatable due to the complexity of the language. 

The final approach is to create our own language. This language should be simple, but able to execute any computation, and abstract enough to be easy to understand. The solution is to implement a language in the LISP family, a concept similar to WebAssembly's Text Format \parencite{wasmtext}. LISP \parencite{mccarthy1960recursive} is a simple recursive language whose operations are based on symbolic expressions. Expressions are lists of elements, where the first element is the operator and the rest of elements the operand. Elements can be atomic (numbers, operators, etc.), or other lists, which will be synthesized into atomic elements by the recursive pattern. By implementing in the language the basic operations described in \reqref[requirement]{SR-FC-02}, we fulfill all our goals for a simple language that supports any possible \gls{ISA}. 


\subsection{Interpreter}\label{subsec:interpreter-study}
\reqref[Requirement]{SR-FC-06} states that the system must be able to execute the \gls{instruction} definition language, and \reqref[requirement]{SR-FC-07} states it should be able to do it \gls{instruction} by instruction. Therefore, the best approach is to create an \gls{interpreter} for our LISP-like language.

To implement an \gls{interpreter} for any language, there are two main steps involved: processing the input into an \gls{AST} (\textit{parsing}), and executing it with the help of the \gls{AST} (\textit{interpreting}) \parencite{mogensen2009basics}.

Nowadays, there are tools that, given a language definition as an input, generate parsers for those languages, simplifying the process of creating an \gls{interpreter}. Such is the case of Lex and yacc \parencite{LevineJohnR1992Ly} (and its modern counterpart Flex and bison \parencite{LevineJohnR2009Fb}), and ANTLR \parencite{ParrTerenceTerenceJohn2010Lip}%, the three of them being the `state of the art' options \parencite{OrtinFrancisco2022Aeeo}
. These parser generator tools generate libraries that parse the inputs into \glspl{AST} that are later used in the execution. The main problem with this approach is that you need to recompile the parser each time you modify the language, which gives less flexibility to add new features (floating point operations, etc.).

Another approach is to build the \gls{interpreter} from scratch. Being a LISP-like language comes with the advantage of simplifying the \gls{interpreter} design. Make-A-Lisp \parencite{mal} is a guide for implementing a generic LISP \gls{interpreter}, and outlines the basic functionalities and architecture.


\subsection{ISA definition format}\label{subsec:isa-format-study}
\reqref[Requirement]{SR-NF-07} specifies that the format used to define and store the \gls{ISA} must be simple and easy to modify. The best approach is to use a text-based format, as they are human-readable and can be modified by any basic text editor, without the need for any specific tools. Furthermore, it allows us to encode with text  both the defined \glspl{assembly} and \gls{instruction} definition language, simplifying the system's implementation.

The best approach is to use a structured data format. There are many formats, but the most widely used are those who store data in attribute-value pairs and arrays, as is the case of the \gls{json} \parencite{ISO21778} and the \gls{yaml} \parencite{yamlspec} formats. While both formats offer libraries for parsing and validating files in many programming languages, the \gls{json} format is a `friendlier' format for \gls{HTTP} due to the fact that it ignores whitespace, unlike \gls{yaml}. Using \gls{json} over \gls{yaml} offers many possibilities for the use of the system in other environments, such as \glspl{web service}.


\subsection{Programming language}\label{subsec:language-study}
As per \reqref[requirements]{SR-NF-01}, the \gls{programming language} must allow the system to be executed natively and on multiple platforms. Native execution has the advantage of producing more performant code, as there is no overhead derived from virtual machines or \glspl{interpreter}. Popular natively compiled languages include C \parencite{KernighanBrianW1988TCpl}, C++ \parencite{StroustrupBjarne2013TCpl}, Rust \parencite{KlabnikSteve2023TRpl}, and Go \parencite{DonovanAlan2015TGPL} \parencite{statista2023langstats}.

Furthermore, \reqref[requirement]{SR-NF-02} states that the system shall also support multiple platforms, including browsers. Through the use of WebAssembly \parencite{HaasAndreas2017Btwu}, most programming languages can be compiled and executed `natively' in the browser, including all previously mentioned languages.

From the mentioned languages, C++ (specifically the current standard at the time of writing, ISO C++20 \parencite{ISOcpp20}) was chosen for several reasons:
\begin{itemize}[nosep]
  \item It's a mature language, with support for modern features and multiple \glspl{programming paradigm}.
  \item It's strongly typed, memory safe, and high-performant, with zero-cost abstractions.  % TODO? get ref from stroustroup
  \item It's an ISO standard language.
\end{itemize}



\section{System architecture}\label{sec:architecture}
The system's architecture can be divided in four main elements:
\begin{itemize}
  \item \textbf{Compiler:} Translates the \glsdisp{assembly}{assembly-like} programs into the \gls{interpreter}'s language, using the \gls{instruction} definitions.
  \item \textbf{Memory:} Stores the \glspl{instruction} and program memory.
  \item \textbf{\gls{CPU}:} Fetches, decodes, and executes the \glspl{instruction} (in the interpreter's language), and stores the registers.
  \item \textbf{\gls{CLI}:} Provides the \gls{UI} and coordinates all the other elements.
\end{itemize}

Figure \figref{component_diagram} details the UML component diagram \parencite{Cook2017} of the system, including all of its components. \tabref{comp-template} provides the template used for the specification of the components, including the description of each attribute. The successive tables describe each of the components.

\graphicfigure[.95]{component_diagram}{Component diagram of the proposed solution}

\printcomptemplate{Identifier}


\begin{component}{Compiler}
  {\glsdisp{compilation}{Compiles} the program to execute into the interpreter's language}  % role
  {Memory subsystem (\reqref{Data Memory}, \reqref{Text Memory}), CPU subsystem (\reqref{Register File}, \reqref{ALU}, \reqref{Control Unit}), \reqref{CLI}}  % dependencias
  {Program to compile, \gls{ISA} definition}  % in-data
  {\reqref{Data Memory} words to store (as defined in the program), program to execute.}  % out-data
  {SR-FC-06}  % origen
  % descripción
  The \glsdisp{compilation}{compiler} translates the \glsdisp{assembly}{assembly-like} program to execute into the interpreter's language, according to the \gls{ISA}, loading the \gls{text memory}. It also preloads the \gls{data memory} segment if defined in the program.
\end{component}

\begin{component}{Data Memory}
  {Stores the program's data}  % role
  {\NA}  % dependencias
  {Memory \glspl{word} to store.}  % in-data
  {Memory \glspl{word} to provide, state.}  % out-data
  {SR-FC-01, SR-FC-04, SR-FC-08}  % origen
  % descripción
  \Gls{data memory} stores \glspl{word} and maps them to specific addresses. It allows reading and writing to those bits, given a specified address. The start and end address is configurable.
\end{component}

\begin{component}{Text Memory}
  {Stores the program's instructions}  % role
  {\NA}  % dependencias
  {Instructions to store, setup configuration.}  % in-data
  {Instructions to provide, state.}  % out-data
  {SR-FC-01, SR-FC-07}  % origen
  % descripción
  \Gls{text memory} stores the instructions to be executed by the \reqref{Control Unit}. Each instruction is composed of an address and the instruction data itself. This component allows for adding new instructions and reading any instruction, given its address.
\end{component}

\begin{component}{Register File}
  {Stores the \gls{register}'s data}  % role
  {\NA}  % dependencias
  {\Gls{register} data to store, setup configuration}  % in-data
  {\Gls{register} data to provide, state, environment (register names).}  % out-data
  {SR-FC-01, SR-FC-04, SR-FC-08}  % origen
  % descripción
  The \gls{register} file holds all the \glspl{register} of the \gls{ISA}, and its contents, \glspl{word}. It also holds the \gls{program counter}. The number of registers is configurable.
\end{component}

\begin{component}{ALU}
  {Executes the operations in the instructions}  % role
  {\reqref{Data Memory}, \reqref{Register File} words}  % dependencias
  {\reqref{Data Memory} words to read, \reqref{Register File} words to read, environment (available operations).}  % in-data
  {\reqref{Data Memory} words to write, \reqref{Register File} words to write, operation to execute.}  % out-data
  {SR-FC-01, SR-FC-02}  % origen
  % descripción
  The \gls{ALU} interacts with the \gls{data memory}, \glspl{register} and \gls{I/O} in order to execute the operations required by the instructions.
\end{component}

\begin{component}{Control Unit}
  {Decodes and executes the instructions}  % role
  {\reqref{ALU}, Memory subsystem (\reqref{Data Memory}, \reqref{Text Memory})}  % dependencias
  {\reqref{Text Memory} instructions, \reqref{CLI} signal, \reqref{ALU} environment, \reqref{Register File} environment.}  % in-data
  {Operation to execute.}  % out-data
  {SR-FC-05, SR-FC-07, SR-FC-10, SR-FC-12, SR-FC-13}  % origen
  % descripción
\end{component}

\begin{component}{CLI}
  {Interfaces with the user and coordinates the system}  % role
  {Memory subsystem (\reqref{Data Memory}, \reqref{Text Memory})}  % dependencias
  {\reqref{Register File} state, \reqref{Data Memory} state, \reqref{Text Memory} state.}  % in-data
  {\reqref{Control Unit} signals, \reqref{Data Memory} setup configuration, \reqref{Register File} setup configuration, program to execute, \gls{ISA}.}  % out-data
  {SR-FC-03, SR-FC-04, SR-FC-07, SR-FC-08, SR-FC-09, SR-FC-10, SR-FC-11, SR-FC-12, SR-FC-13}  % origen
  % descripción
  The \gls{CLI} is the main component of the system. It's in charge of executing its main functionalities: validating the input files, executing the program (by signaling the \reqref{Control Unit}) and viewing the state of the memory and register file.
\end{component}

\FloatBarrier


\subsubsection{Traceability}\label{subsec:comp-traceability}
A traceability matrix was generated (\tabref{traceability-comp-fc}) in order to validate that all the software requirements, defined in Subsection \ref{subsec:soft-requirements}, were satisfied in the components of the architecture.

% components - SR-FC
\traceabilityTable{traceability-comp-fc}
  {^[^US][^R][^-]}{^SR\37-FC}
  {Traceability between functional requirements and components}
% Note that the `-` is a special character in Lua patterns and should be escaped, using \37.

\FloatBarrier


\subsection{Instruction definition language}\label{subsec:inst-lang-design}
As stated in \subsecref{inst-lang-study}, the language used to define the instructions is a language from the LISP family. The developed language is LUISP-DA (\textit{assembLy analogoUs lISP DiAlect}).

Being a LISP-like language, instructions are defined as LISP expressions, comprised of lists of symbols and/or other lists, delimited by parenthesis. Inner expressions are evaluated first, recursively. To simplify the language, there are no variables, macros, or functions, as they are not necessary for defining basic instructions.

\noindent
There are three data types:
\begin{itemize}
  \item \qdist{NUM}, integer numbers.
  \item \qdist{BOOL}, boolean values.
  \item \qdist{NIL}, `none' type.
\end{itemize}

The first element of each expression is called the \textit{operator}, and takes the rest of the elements of the expression the values the operator is applied to. There are several predefined operators, which can be categorized by type they evaluate to:
\begin{itemize}
  \item Operators that evaluate to \qdist{NUM}:
  \begin{itemize}
    \item \textit{add}: Adds two values.
    \item \textit{sub}: Subtracts two values.
    \item \textit{neg}: Negates a value.
    \item \textit{mul}: Multiplies two values.
    \item \textit{div}: Divides two values.
    \item \textit{mod}: Module of two values.
    \item \textit{get-reg}: Gets the contents of a register.
    \item \textit{set-reg}: Stores a value in a register. Evaluates to the value.
    \item \textit{get-pc}: Gets the current \gls{program counter}.
    \item \textit{set-pc}: Stores a value in the \gls{program counter}. Evaluates to the value.
    \item \textit{get-mem}: Gets the value in a memory address.
    \item \textit{set-mem}: Stores a value in a memory address. Evaluates to the value.
  \end{itemize}
  \item Operators that evaluate to \qdist{BOOL}:
  \begin{itemize}
    \item \textit{lt}: Evaluates if a value is less than other value.
    \item \textit{le}: Evaluates if a value is less or equal than other value.
    \item \textit{gt}: Evaluates if a value is greater than other value.
    \item \textit{ge}: Evaluates if a value is greater or equal than other value.
    \item \textit{eq}: Evaluates if a value is equal to other value.
    \item \textit{ne}: Evaluates if a value is not equal to other value.
    \item \textit{not}: Negates a value.
  \end{itemize}
\end{itemize}

\noindent
There are other `special' operators, as well:
\begin{itemize}
 \item Blocks (\qdist{block}) allow the user to evaluate an infinite number of expressions, one after the other, the whole expression evaluating to \qdist{NIL}.
 \item Conditionals (\qdist{conditional}) enable \glsdisp{branch}{branching}, by using an \textit{if-then-else} structure. If the first element evaluates to \texttt{true}, the the conditional evaluates to the second element; and if the condition evaluates to \texttt{false} (or \qdist{NIL}), it evaluates to the third element instead (if the third element doesn't exist, it evaluates to \qdist{NIL}).
 \item System calls (\qdist{call}) evaluates the first element, known as the \textit{opcode}, and executes the specified \textit{call} with its arguments. The \textit{opcodes} are defined in the \gls{ISA}. There defined \textit{calls} are:
 \begin{itemize}[nosep]
    \item \textit{print-int}: Prints a value as an integer to \gls{STDOUT}.
    \item \textit{read-int}: Stores a value from \gls{STDIN} in a register.
    \item \textit{print-char}: Prints a value as an ASCII character to \gls{STDOUT}.
    \item \textit{read-char}: Stores an ASCII character from \gls{STDIN} in a register.
    \item \textit{exit}: Ends the execution of the program.
  \end{itemize}
\end{itemize}

\noindent
The full grammar's representation, in Backus-Naur Form \parencite{Backus1963Rev}, is the following:
%
\begin{grammar}
  <expression> ::= <ATOMIC>
    \alt <b-expression>
    \alt <i-expression>
    \alt <block>
    \alt <conditional>
    \alt <call>

  % boolean expression
  <b-expression> ::= <BOOL>
    \alt <L>`<' <i-expression> <i-expression><R>\hfill\textit{lt}
    \alt <L>`<=' <i-expression> <i-expression><R>\hfill\textit{le}
    \alt <L>`>' <i-expression> <i-expression><R>\hfill\textit{gt}
    \alt <L>`>=' <i-expression> <i-expression><R>\hfill\textit{ge}
    \alt <L>`==' <i-expression> <i-expression><R>\hfill\textit{eq}
    \alt <L>`!=' <i-expression> <i-expression><R>\hfill\textit{ne}
    \alt <L>`!' <b-expression><L>\hfill\textit{not}

  % integer expression
  <i-expression> ::= <NUM>
    \alt <L>`+' <i-expression> <i-expression><R>\hfill\textit{add}
    \alt <L>`-' <i-expression> <i-expression><R>\hfill\textit{sub}
    \alt <L>`-' <i-expression><R>\hfill\textit{neg}
    \alt <L>`*' <i-expression> <i-expression><R>\hfill\textit{mul}
    \alt <L>`/' <i-expression> <i-expression><R>\hfill\textit{div}
    \alt <L>`%' <i-expression> <i-expression><R>\hfill\textit{mod}
    \alt <L>`reg' <REG><R>\hfill\textit{get-reg}
    \alt <L>`reg!' <REG> <i-expression><R>\hfill\textit{set-reg}
    \alt <L>`pc'\hfill\textit{get-pc}
    \alt <L>`pc!' <i-expression><R>\hfill\textit{set-pc}
    \alt <L>`mem' <i-expression><R>\hfill\textit{get-mem}
    \alt <L>`mem!' <i-expression> <i-expression><R>\hfill\textit{set-mem}

  <block> ::= <L>`do' <expr-block><R>

  <expr-block> ::= <expression>
    \alt <expression> <expr-block>

  <conditional> ::= <L>`if' <b-expression> <expression> <expression><R>

  <call> ::= <L>`call' <NUM> <i-expression><R>\hfill\textit{print-int / print-char}
    \alt <L>`call' <NUM> <REG><R>\hfill\textit{read-int / read-char}
    \alt <L>`call' <NUM><R>\hfill\textit{exit}

  <L> ::= `('

  <R> ::= `)'

  <ATOMIC> ::= <NUM>
    \alt <BOOL>
    \alt <NIL>

  <NUM> ::= <DEC>
    \alt `0x'<HEX>

  <DEC> ::= <I-DIGIT>
    \alt <I-DIGIT><DEC>

  <HEX> ::= <H-DIGIT>
    \alt <H-DIGIT><HEX>

  <I-DIGIT> ::= `0' | `1' | `2' | `3' | `4' | `5' | `6' | `7' | `8' | `9'

  <H-DIGIT> ::= <I-DIGIT> | `a' | `b' | `c' | `d' | `e' | `f'

  <BOOL> ::= `true'
    \alt `false'

  <NIL> ::= `nil'

  <REG> ::= \textit{The set of register names defined in the \gls{ISA}}
\end{grammar}


\noindent
For more information about the use of the language, see \appref{manual}.


\subsection{Control Unit}\label{subsec:interpreter-design}
The LUISP-DA \gls{interpreter} is implemented in the \reqref{Control Unit} component. As stated in \subsecref{interpreter-study}, the architecture of the \gls{interpreter} will be based on the Make-A-Lisp \parencite{mal} architecture. Nevertheless, as we only need to model a simple \glsdisp{assembly}{assembly-like language}, we'll use a subset of the original architecture, and modify it for our purposes.

\figref{interpreter_architecture} shows an overview of the LUISP-DA interpreter. As stated in Subsection \ref{subsec:interpreter-study}, the interpretation is done in two stages:
\begin{enumerate}
  \item \textbf{\nameref{subsubsec:read}:} The instruction, a string, is parsed into an \gls{AST}.
  \item \textbf{\nameref{subsubsec:eval}:} The \gls{AST} is evaluated recursively.
\end{enumerate}

Finally, after each instruction, the \gls{program counter} is incremented.

\graphicfigure[.8]{interpreter_architecture}{Interpreter architecture}


\subsubsection{READ}\label{subsubsec:read}
In this stage, the \gls{interpreter} first reads the instruction from the \reqref{Text Memory} (\textit{read-inst}), then parses the string (\textit{read-str}) and extracts (and stores) the tokens (\textit{tokenize}). After that, it starts generating the \gls{AST} by reading and evaluating the first token. This is done through the use of three functions:
\begin{itemize}
  \item \textit{read-token}: Checks if the current token is the start of an \qdist{L} token. If it is, it calls \textit{read-list} on the next token; otherwise it calls \textit{read-atom} on the current token. It returns the generated \gls{AST}.
  \item \textit{read-list}: Creates a new \gls{AST} node, and iteratively calls \textit{read-token} on all the following tokens until it reaches an \qdist{R} token, adding the resulting nodes as children of the current node. Finally, it returns the new node.
  \item \textit{read-atom}: Evaluates the current token's type (by searching the token in the ENV), and returns a new \gls{AST} node.
\end{itemize}


\subsubsection{EVAL}\label{subsubsec:eval}
After \textit{read-inst} returns the \gls{AST} of the instruction, it is evaluated through the use of two functions:
\begin{itemize}
  \item \textit{eval-token}: Reads the node's token. If it's an \qdist{R} token, it calls \textit{eval-ast} on all the arguments (all children nodes except the first one), executes on the \reqref{ALU} the operator (first child) with the results as arguments, and returns the result. If it has no children, it returns a \qdist{NIL} node, and if it has only one child, it returns the child node. In any other case, it calls \textit{eval-ast} on the node.
  \item \textit{eval-ast}: Reads the node's token. If it's an \qdist{R} token, it creates a new node with an \qdist{R} token, adds as children of that node the result of calling \textit{eval-token} on all the original node's children, and returns the new node. Otherwise, it returns the current node.
\end{itemize}

\noindent
The first symbol of the \gls{AST} is passed to \textit{eval-token} to start the evaluation.


\subsection{Compiler}\label{subsec:compiler-design}
% .data, .text



\subsection{ISA definition}\label{subsec:isa-design}
% JSON -> diagrama entidad-relación?


\subsection{User interface}\label{subsec:ui-design}
% CLI -> sequence diagram


