\chapter{State of the Art}\label{chap:state-of-the-art}
This chapter presents the current stage of the technologies related to our application, specifically the different kinds of simulators that are currently present in the market (\sectionref{sec:assembly-simulators}), both specific to one architecture (\subsectionref{subsec:specific-assembly-simulators}) and architecture-agnostic simulators (\subsectionref{subsec:generic-assembly-simulators}).\\
Finally, it includes a comparison between all the mentioned simulators and our proposed application (\sectionref{sec:comparison}).



\section{Assembly simulators}\label{sec:assembly-simulators}
An `\gls{assembly simulator}' is a CPU simulator that enables the user to program it through an \gls{assembly}, and it usually has an educational purpose: either to aid the user in learning the language, or as an exercise for the programmer in order to give them a better understanding of CPU architecture and software development.

These simulators typically offer an interface that allows the user to execute the program step-by-step and see the current state of the simulated \gls{computer}.


\subsection{Specific simulators}\label{subsec:specific-assembly-simulators}
The vast majority of simulators that can be found today focus on emulating a specific \gls{ISA}, which can range from simple 8-bit microprocessors like the Intel 8080\supercite{i8080emulator} to architectures that are used today, like \gls{ARM} Thumb\supercite{QtARMSim}.

\noindent
Here, we'll focus on two examples: Kite\supercite{song_kite2019} and ARMLite\supercite{ARMLite}.


\subsubsection*{Kite}  % Kite: https://github.com/yonseicasl/Kite
Kite\supercite{song_kite2019} is a simulator that models a five-stage \gls{pipeline} \gls{RISC-V} CPU, based on the model described in \textit{Computer organization and design: the hardware/software interface (RISC-V edition)}\supercite{PattersonDavidA.2018Coad}, by D. Patterson and J. Hennessy, and implemented in C/C++. It was developed in 2019 in order to provide students of Yonsei University (Seoul, Korea) with an easy-to use simulator to accompany follow its Computer Architecture course.

It incorporates advanced features derived from a \gls{pipeline} architecture, such as \gls{instruction dependency} detection and \gls{pipeline stalls}, among others. These `under the hood' features offer a better understanding of the underlying concepts of computer architecture.

The simulator consists of an executable with a \gls{CLI} that takes three input files: the program's code, the register's state, and the data memory's state. It loads the state of registers and data memory, executes the program's code and saves the state to the specified files, after the execution, printing some statistics (Figure \ref{fig:kite}). It also implements a \gls{debug} mode that prints the state of each instruction on the \gls{pipeline}, each \gls{clock cycle}.

\begin{figure}[h]
  \caption{Kite simulator \gls{CLI}.}
  \includegraphics[width=0.7\textwidth]{kite.png}
  \label{fig:kite}
\end{figure}

Its source code can be found in the Yonsei University Computer Architecture and Systems Lab's GitHub\supercite{kiteGH}.

% \subsubsection*{MARS}  % https://courses.missouristate.edu/KenVollmar/MARS/index.htm
% MARS\supercite{VollmarKenneth2006MaeM} follows the \gls{MIPS} architecture, another \gls{RISC} type architecture.

\subsection*{ARMLite}  % https://www.peterhigginson.co.uk/ARMlite/
ARMLite\supercite{ARMLite} is a web-based simulator for a 32-bit `\glsdisp{ARM}{ARM-like}' processor. It includes a basic set of instructions described in \textit{Assembly Language Programming}\supercite{PawsonRichard.2020Ass}, by R. Pawson. This simulator was developed for educational purposes, but specifically to target AQA\supercite{AQA}'s Assembly language instruction set for its A-level computer syllabus\supercite{AQAInstructionSet}.

The simulator offers a \gls{GUI} (Figure \ref{fig:armlite}) that shows the current state of the registers and the data memory, as well as the current instruction and the status bits. It also adds \gls{I/O} support (a text box and a display), and options to start, stop, pause, resume, and slow or speed up execution, as well as step-by-step execution. It allows for users to load, save, and edit their programs, as well as a box to output system information (errors, last action performed, etc.).

\begin{figure}[h]
  \caption{ARMLite executing Conway's Game of Life\normalfont\supercite{Gardner1970fantastic}.}
  \includegraphics[width=0.7\textwidth]{ARMLite.png}
  \label{fig:armlite}
\end{figure}

While not being \gls{FOSS} \textit{per se} (we couldn't find any information about the license used by this simulator), the simulator uses JavaScript, which means the browser interprets the webpages's source code locally. As the code is \glsdisp{obfuscate}{unobfuscated} and is, to date, public and free to use, we \textit{could} consider this simulator \gls{FOSS}.


\subsection{Generic simulators}\label{subsec:generic-assembly-simulators}
We consider simulators that are able to define and execute multiple \glspl{ISA} `generic simulators'. They add a significant complexity to those types of simulators, due to the fact that they must be able to dynamically adapt to the different architectures.

\noindent
Two different approaches to creating such simulators can be found in two of them: CREATOR\supercite{creatorZenodo} and Sail\supercite{ArmstrongAlasdair2019IsfA}.


% sail
\subsubsection*{Sail}  % https://github.com/rems-project/sail/
Sail\supercite{ArmstrongAlasdair2019IsfA} is an \gls{ISA} definition language. It was developed by the Rigorous Engineering of Mainstream Systems (REMS)\supercite{rems} research group, from the University of Cambridge, as a tool to allow \glspl{ISA} to be mathematically modeled and verified and formally proven.

This tool can not only type-check instruction and vector lengths for the \gls{ISA}, generate \glspl{theorem prover} definitions of the architecture and tests, or automatically generate documentation, but, more importantly for our specific case, it can generate executable simulators in C or OCaml based on those \glspl{ISA}.




\subsubsection*{CREATOR}  % https://creatorsim.github.io/





\section{Comparison}\label{sec:comparison}


\begin{table}[h]
  \caption{Feature comparison of current assembly simulators.}
  \tiny  % too big to use normal font
  \resizebox{\textwidth}{!}{  % fit to textwidth
    % \begin{tabular}{>{\bfseries}p{3cm}|M{1.7cm}M{1.7cm}M{1.7cm}M{1.7cm}M{1.7cm}M{1.7cm}}
    % \begin{tabular}{>{\bfseries}m{3cm}cccccc}
    \begin{tabular}{>{\bfseries}lccccc}
      \hline
      Simulator   & Kite       & ARMLite    & CREATOR    & Sail       & Proposal\\
      \hline
      Language    & C/C++      & JavaScript & JavaScript & OCaml      & C++23\\
      % or LICENSE type
      \gls{FOSS}  & \checkmark & \checkmark\tablefootnote{No license was found.}
                                            & \checkmark & \checkmark & \checkmark\\
      Architecture definition file 
                  &            &            & \checkmark & \checkmark\tablefootnote{Simulator needs to be \glsdisp{compilation}{recompiled} for any new architecture.} 
                                                                      & \checkmark\\
      % Execute files
      %             & \checkmark & \checkmark & \checkmark & \checkmark & \checkmark\\
      \gls{CLI}   & \checkmark &            & \checkmark & \checkmark & \checkmark\\
      \gls{I/O}   &            & \checkmark & \checkmark & \checkmark & \checkmark\\
      Step-by-step execution
                  &            &            & \checkmark & \checkmark & \checkmark\\
      Simple \& generic language 
                  &            &            &            &            & \checkmark\\
      \hline
    \end{tabular}
  }
\end{table}