\chapter{Project Plan}\label{chap:planning}
This chapter presents an overview of the development and logistics of the project. The planning of the project (\secref{planning}) is detailed, analyzing its budget and overall cost (\secref{budget}). Furthermore, it analyzes the different legislation and regulations that may apply to the project (\secref{regulation}), and discusses the socio-economic environment in which it was carried out. (\secref{environment}).


\section{Planning}\label{sec:planning}
This section details the project's planning, by describing the followed methodology and detailing the duration of each part.

\subsection{Methodology}
Due to the characteristics of the design, the development process was divided into four iterations:
\begin{enumerate}[label=\Roman*.]
  \item \textbf{\Gls{memory} and \Gls{register} File}. This iteration's goal was to implement the \reqref{Text Memory}, \reqref{Data Memory}, and \reqref{Register File} components separately.
  \item \textbf{Control Unit and \gls{ALU}}. This phase consists on implementing the \reqref{Control Unit} and \reqref{ALU} components, and connecting them to all the previous components. The goal of this iteration is to be able to simulate the full \gls{CPU} and \gls{memory}, and execute LUISP-DA instructions.
  \item \textbf{Compiler}. This iteration consists on implementing the \reqref{Compiler} component, with the purpose of being able to load \glspl{ISA} and execute programs.
  \item \textbf{\gls{CLI}}. The final iteration implements the \reqref{CLI} component, adding features such as the \gls{REPL environment} and the whole \gls{UI}.
\end{enumerate}

An iterative methodology was chosen, specifically Bohem's spiral model \parencite{BohemBSpiral}, in order to ensure that, before implementing a component, all components that it depends on are correctly implemented. It also simplifies the development life cycle and makes it more flexible, as it allows the developer to go back and modify previous elements, and encourages prototyping.

The life cycle development process of this model (\figref{spiral_model}) has four phases, which are repeated during the different iterations of the model. These phases are:
\begin{enumerate}
  \item \textbf{Planning}: The user requirements are gathered, and the iteration's objectives are determined.
  \item \textbf{Analysis}: An analysis of the user requirements is performed in order to identify potential risks, and the test cases are designed.
  \item \textbf{Development and testing}: The software is implemented, and tests are performed.
  \item \textbf{Evaluation}: The software is evaluated with the client, in order to provide feedback. In this specific case, the tutor acted as the client. This is the critical task of the life cycle, as the iteration isn't finished until the software is approved.
\end{enumerate}

\svgfigure{spiral_model}{Spiral model life cycle}


\subsection{Time Estimation}\label{subsec:time-estimation}
The time estimation of the project was designed with the use of a Gantt chart \parencite{clark1922gantt} (\figref{gantt}). This diagram shows all the performed tasks in each iteration of the life cycle, plus an extra documentation task in each iteration for the drafting of this thesis. A final `Report' task was added to represent the time spent finishing this report.

The project had a total duration of 9 months, averaging 35 hours per month (taking into account days off, holidays, sick days, and other drawbacks), around 10 hours per week. Therefore, the total time dedicated to this project is 315 hours.


\begin{landscape}
  \begin{figure}
    \ffigbox[\FBwidth]
      {%
        \caption{Gantt chart}
        \label{fig:gantt}
      }%
      {
        \begin{ganttchart}[
          hgrid,
          % vgrid,
          time slot format=isodate,
          time slot unit=day,
          x unit=.075cm,
          y unit chart=.481cm,
          y unit title=.7cm,
          title label font=\footnotesize,
          group label font=\bf\scriptsize,
          bar label font=\scriptsize,
          milestone label font=\it\scriptsize,
          % expand chart=\textwidth
        ]{2023-09-13}{2024-06-13}

          \gantttitlecalendar{year, month=shortname} \\

          \ganttgroup{I. \Gls{memory} and \Gls{register} File}{2023-09-13}{2023-10-21} \\
          \ganttbar{Planning}{2023-09-13}{2023-09-15}\\
          \ganttlinkedbar{Analysis}{2023-09-16}{2023-09-18}\\
          \ganttlinkedbar{Development and testing}{2023-09-18}{2023-10-12}\\
          \ganttlinkedbar{Evaluation}{2023-10-12}{2023-10-14}\\
          \ganttlinkedbar{Documentation}{2023-10-14}{2023-10-21}\\

          \ganttgroup{II. Control Unit and \gls{ALU}}{2023-10-22}{2023-12-23} \\
          \ganttbar{Planning}{2023-10-22}{2023-10-24}\\
          \ganttlinkedbar{Analysis}{2023-10-25}{2023-10-27}\\
          \ganttlinkedbar{Development and testing}{2023-10-27}{2023-12-10}\\
          \ganttlinkedbar{Evaluation}{2023-12-10}{2023-12-14}\\
          \ganttlinkedbar{Documentation}{2023-12-14}{2023-12-23}\\

          \ganttgroup{III. Compiler}{2024-01-07}{2024-03-22} \\
          \ganttbar{Planning}{2024-01-07}{2024-01-09}\\
          \ganttlinkedbar{Analysis}{2024-01-10}{2024-01-12}\\
          \ganttlinkedbar{Development and testing}{2024-01-13}{2024-02-25}\\
          \ganttlinkedbar{Evaluation}{2024-02-26}{2024-02-28}\\
          \ganttlinkedbar{Documentation}{2024-02-29}{2024-03-22}\\

          \ganttgroup{IV. \gls{CLI}}{2024-04-02}{2024-05-26} \\
          \ganttbar{Planning}{2024-04-02}{2024-04-04}\\
          \ganttlinkedbar{Analysis}{2024-04-05}{2024-04-07}\\
          \ganttlinkedbar{Development and testing}{2024-04-08}{2024-05-01}\\
          \ganttlinkedbar{Evaluation}{2024-05-02}{2024-05-05}\\
          \ganttlinkedbar{Documentation}{2024-05-06}{2024-05-26}\\

          \ganttgroup{Report}{2024-05-27}{2024-06-12} \\

          % \ganttmilestone{Project delivery}{2024-06-12}

          % link iterations
          \ganttlink{elem0}{elem6}
          \ganttlink{elem6}{elem12}
          \ganttlink{elem12}{elem18}
          \ganttlink{elem18}{elem24}
          % \ganttlink{elem24}{elem25}
        \end{ganttchart}
      }
  \end{figure}
\end{landscape}



\section{Budget}\label{sec:budget}
This section details the project's budget, based on the time estimation and planning described in \secref{planning}. \tabref{project-info} summarizes the main characteristics of the project, including the total budget.

\makeatletter

\begin{table}[htb]
  \ttabbox[\FBwidth]
    {%
      \caption{Project information}
      \label{tab:project-info}
    }%
    {
      \begin{tabular}{>{\bfseries}p{3.5cm}p{9cm}}
        \toprule
        Title               & \textit{\@title} \\ \midrule
        Author              & \@author \\ \midrule
        Department          & Departamento de Informática \\ \midrule
        Start date          & 13th of September of 2023 \\ \midrule
        End date            & 13th of June of 2024 \\ \midrule
        Duration            & 9 months \\ \midrule
        Total budget        & 12.161,84 € \\
        \bottomrule
      \end{tabular}
    }
\end{table}

\makeatother


The different costs will be divided in two parts: direct costs (those associated with personnel and equipment), and indirect costs (those with an indirect influence to the project). These costs don't include taxes, which will be included in the costs summary on Subsection \ref{subsec:cost-summary}.


\subsection{Direct Costs}
Direct costs are those that are directly related with the development of the project. These can be divided into two groups:
\begin{itemize}
  \item \textbf{Personnel costs}: These vary in relation to the qualifications, experience, and geographical location of each member.
  \item \textbf{Equipment costs}: These are all the costs associated to the tools required for developing the software, mainly hardware and software tools.
\end{itemize}

\noindent
Personnel costs can be divided between four different roles:
\begin{itemize}
  \item \textbf{Project manager}: Manages the project's schedule and provides feedback.
  \item \textbf{Analyst}: Analyzes the user's requirements, designs the architecture of the system, and writes documentation.
  \item \textbf{Programmer}: Implements the required functionalities.
  \item \textbf{Tester}: Designs and performs the tests for the different functionalities.
\end{itemize}

The tutor played the role of project manager, while the student played the rest of the roles. \tabref{personnel-cost} shows the direct costs caused by each of the roles, and the total personnel costs.

\begin{table}[htb]
  \ttabbox[\FBwidth]
    {%
      \caption{Personnel cost}
      \label{tab:personnel-cost}
    }%
    {
      \begin{tabular}{lrrr}
        \toprule
        \textbf{Role} & \textbf{Hours} & \textbf{Cost per hour} & \textbf{Total} \\
        \midrule
        Project Manager &  40 h & 65,00 €/h & 2.600,00 € \\
        Analyst         &  80 h & 35,00 €/h & 2.800,00 € \\
        Programmer      & 175 h & 30,00 €/h & 5.250,00 € \\
        Tester          &  60 h & 20,00 €/h & 1.200,00 € \\
        \midrule
        \textbf{Total}  & 385 h &           & \textbf{11.850,00 €} \\
        \bottomrule
      \end{tabular}
    }
\end{table}

Equipment costs refer to those caused by equipment acquisition and usage. For software, all the software tools used for this project were \gls{FOSS}. The chargeable cost, $C$, for each item is calculated using Equation \eqref{eq:chargeable-cost}, where:
\begin{itemize}
  \item $c$ is the cost of the item.
  \item $d$ is the time the item has been used.
  \item $u$ is the percentage of the total time the item was used for the project.
  \item $D$ is the item's depreciation period.
\end{itemize}

\begin{equation}\label{eq:chargeable-cost}
  C = \frac{c \cdot u \cdot d}{D}
\end{equation}

\tabref{equipment-cost} shows the chargeable costs of each equipment item, and the total equipment costs.

\begin{table}[htb]
  \ttabbox[\FBwidth]
    {%
      \caption{Equipment cost}
      \label{tab:equipment-cost}
    }%
    {
      \begin{adjustbox}{max width=\textwidth}  % fit to textwidth
        % \begin{tabular}{lR{2cm}R{2cm}R{2.3cm}R{2.3cm}R{2.3cm}}
        \begin{tabular}{lrrrrr}
          \toprule
          \textbf{Item} & \textbf{Cost ($c$)} & \textbf{Usage ($u$)} & \textbf{Dedication ($d$)} & \textbf{Depreciation ($D$)} & \textbf{Chargeable cost ($C$)}   \\
          \midrule
          Laptop      & 649,00 € & 60 \% & 9 months &  72 months &  48,68 € \\
          Monitor     & 124,99 € & 40 \% & 9 months &  60 months &  74,99 € \\
          Mouse       &  79,95 € & 50 \% & 9 months &  36 months &   9,99 € \\
          Adapter     &  39,90 € & 70 \% & 9 months &  48 months &   5,24 € \\
          Video cable &   5,75 € & 50 \% & 9 months &  48 months &   0,54 € \\
          Software    &   0,00 € & 40 \% & 9 months & 120 months &   0,00 € \\
          \midrule
          \textbf{Total}
                      & 899,59 € &       &          &           &\textbf{139,44 €} \\
          \bottomrule
        \end{tabular}
      \end{adjustbox}
    }
\end{table}


\subsection{Indirect Costs}
Indirect costs are those that are present during the development process, but cannot be assigned directly to any product.

For the energy consumption, it is assumed that the laptop's energy usage averages 80W, and the monitor and mouse average 20W, and that they were used during the whole development process, totaling 360h. Therefore, the total energy consumed is $100W \cdot 360 h = 3.600 Wh$. The internet plan includes a 600mb optic fiber connection, 31,80 €/month, which is shared between three people, two of them not included in the project, therefore the cost applicable to the project is a third of that.


\begin{table}[htb]
  \ttabbox[\FBwidth]
    {%
      \caption{Indirect costs}
      \label{tab:indirect-cost}
    }%
    {
      \begin{tabular}{lrrr}
        \toprule
        \textbf{Resource} & \textbf{Unitary cost} & \textbf{Units} & \textbf{Total} \\
        \midrule
        Electricity    &     0.14 €/kWh & 3.600 Wh &  0,50 € \\
        Internet       &  10,60 €/month & 9 months & 95,40 € \\
        Transportation &      8 €/month & 9 months & 72,00 € \\
        \midrule
        \textbf{Total} & & & \textbf{172,40 €} \\
        \bottomrule
      \end{tabular}
    }
\end{table}


\subsection{Costs Summary}\label{subsec:cost-summary}
\tabref{cost-summary} includes a summary of the costs of the whole project.

\begin{table}[htb]
  \ttabbox[\FBwidth]
    {%
      \caption{Costs summary}
      \label{tab:cost-summary}
    }%
    {
      \begin{tabular}{p{5cm}r}
        \toprule
        Personnel          & 11.850,00 € \\
        Equipment          &    139,44 € \\
        Indirect costs     &    172,40 € \\
        \midrule
        \textbf{Total}     & \textbf{12.161,84 €} \\
        \bottomrule
      \end{tabular}
    }
\end{table}

Therefore, the total cost of the project is \textbf{12.161,84 € (TWELVE THOUSAND ONE HUNDRED SIXTY-ONE EURO AND EIGHTY-FOUR CENTS)}.


\subsection{Project Offer Proposal}\label{subsec:offer-proposal}
\tabref{offer-proposal} details an offer proposal for the project. This proposal includes the estimated risks, expected benefits, and taxes. After applying all these concepts, the final cost of this project, in case it is presented to a third-party client, is \textbf{20.307,84 € (TWENTY THOUSAND THREE HUNDRED SEVEN EURO AND EIGHTY-FOUR CENTS)}.

\begin{table}[htb]
  \ttabbox[\FBwidth]
    {%
      \caption{Offer proposal}
      \label{tab:offer-proposal}
    }%
    {
      \begin{tabular}{lrrr}
        \toprule
        \textbf{Concept} & \textbf{Increment} & \textbf{Partial cost} & \textbf{Aggregated cost} \\
        \midrule
        Project cost   & --    & 12.161,84 € & 12.161,84 € \\
        Risk           & 20 \% &  2.432,37 € & 14.594,21 € \\
        Benefits       & 15 \% &  2.189,13 € & 16.783,34 € \\
        Tax            & 21 \% &  3.524,50 € & 20.307,84 € \\
        \midrule
        \textbf{Total} & 67 \% & & \textbf{20.307,84 €} \\
        \bottomrule
      \end{tabular}
    }
\end{table}



\section{Regulatory Framework}\label{sec:regulation}
This section details and discusses the different regulation that may apply to the project.

\subsection{Applicable Legislation}\label{subsec:legislation}
As the software is executed locally, and it does not transmit or use private data from the users, no data protection laws apply. The software is also used for educational purposes, meaning there are no risks involved in the execution of the software and no regulatory compliance is required.


\subsection{Technical Standards}\label{subsec:standards}
The software makes use of two technical standards:
\begin{itemize}
  \item ISO/IEC 14882:2020(E) \parencite{ISOcpp20}, the standard for the C++20 programming language, which is the one the software is implemented in.
  \item ISO/IEC 21778:2017 \parencite{ISO21778}, the standard for the \gls{json} data format, which is used to store the \gls{ISA} definition files.
\end{itemize}


\subsection{Licenses}\label{subsec:licenses}
The software uses and redistributes one third-party library, \textit{JSON for Modern C++} \parencite{nlohmann/json}. This library is licensed under the MIT License \parencite{MITLicense}, which allows permission to use, copy, modify, or redistribute source code without restrictions or limitations.

The software itself is licensed under the \gls{gplv3} \parencite{gplv3}, in order to allow free use and modification of the source code, while ensuring it will always remain that way. Furthermore, the software is publicly available at \myrepo.



\section{Socio-Economic Environment}\label{sec:environment}
In 2010, RISC-V appears as an open, modular \gls{ISA} \parencite{riscvOrigin}. This comes at a time when the commercial war between the USA and China has just started, affecting the exportation of microchips. This forces countries and corporations to start investing in manufacturing their own chips, and an open standard such as the mentioned RISC-V can act as a base for designing and implementing their own \glspl{ISA}.

Furthermore, recent technological advances in \gls{cloud computing}, \gls{IoT}, and \gls{AI} greatly benefit from hardware acceleration, which pushes for the design of new, more specific \glspl{ISA} that need to be tested and validated.

The recent advancements (and overall push) in computing has also put a great focus in cybersecurity, as our increasing dependence on computing resources creates greater security risks. Many of the vulnerabilities and exploits occur at the \glsdisp{assembly}{assembly} level which, together with the constant search for efficiency, creates a high demand for people with deep knowledge of \gls{assembly} programming.

As discussed in \chapref{state-of-the-art}, there are not many didactic and generic \gls{assembly} simulators that can be used to introduce people to \gls{assembly} programming and \glspl{ISA} design. Our proposal aims to foment these areas of computer science by providing a simple and easy to understand tool for anyone to start exploring different instruction sets, while also aiding people with the sometimes difficult process of learning \gls{assembly} programming.
